\documentclass{article}
\usepackage[utf8]{inputenc}
\usepackage{minted}
\usepackage{hyperref}
\usepackage{graphicx}

\title{STM32 LoRa and Sigfox Discovery kit}
\author{cmonaton }
\date{July 2019}

\begin{document}

\maketitle

\section{Introduction}
Le but de ce tuto est d'envoyer les coordonnées GPS de la carte sur le réseau LoRaWAN.
carte : B-L072Z-LRWAN1
\section{logiciels à installer}

%\subsection{stm32 workbench}
%\subsubsection{installation}
%Se créer un compte sur OPEN STM32 community.
%Télécharger l'installeur pour linux, l'exécuter



\subsection{STM32 cube IDE }
Télécharger le logiciel sur le site de ST


\subsection{Installation de STM32 cube programmer : flasher de code}

Installer java JRE 1.8 pour l'utiliser

\subsection{Application demo}

Demo : https://www.st.com/en/embedded-software/i-cube-lrwan.html


\subsection{Générer un fichier .bin pour la carte depuis le PC}
Avec STM32 Cube IDE. Voir détail au 4 avec l'exmple du programme End Node.



\subsection{Flasher le FW avec STM32 cube programmer}

\begin{enumerate}


      \item Connecter la carte : Parfois la carte ne peut plus se connecter au PC pour flasher le code.
    Explication : le FW bloque le port de STLINK
    Solution : Maintenir appuyé le bouton reset de la carte au moment de la connection avec STM32Cube Programmer

    \item 
    A côté de "device memory" dans un onglet selon l'image, ajouter un fichier et "Download" un fichier .elf ou .bin
    
    
    \begin{figure}[H]
\begin{center}
\advance\leftskip-3cm
\advance\rightskip-3cm
\includegraphics[keepaspectratio=true,scale=0.4]{st_cube_programmer_2.png}
\label{visina8}
\end{center}\end{figure}
    
    
    \item upgrader le firmware pour STlink quand on flash sur une carte différente.
    Télécharger la dernière version sur le site de ST, Version : 2.34.25  : \url{https://www.st.com/content/st_com/en/products/development-tools/software-development-tools/stm32-software-development-tools/stm32-programmers/stsw-link007.html} \\
    Pour linux : lire les readme : 
    \begin{minted}{bash}
    java -jar STLinkUpgrade.jar dans le bon dossier
    et :
    sudo cp *.* /etc/udev/rules.d
    \end{minted}
    dans stsw-link007/AllPlatforms/StlinkRulesFilesForLinux
\end{enumerate}



\section{Utiliser un terminal série pour communiquer avec la carte}


Sur Ubuntu :
Localiser le port USB sur lequel la carte est branchée : 
\begin{enumerate}
\item avec tkdiff
\begin{minted}{bash}
sudo apt-get install tkdiff

cd ~
touch log.txt log2.txt
ls /dev > log.txt
#débrancher la carte
ls /dev > log2.txt
tkdiff log.txt log2.txt
\end{minted}


\item Avec dmesg :


\begin{minted}{bash}
dmesg | grep tty 
\end{minted}{bash}
dans mon cas : ttyACM0: USB ACM device

\end{enumerate}

\subsection{Liaison série avec la carte :}

\begin{enumerate}
\item
Avec Minicom : sudo apt-get install minicom
Configurer la liason : sudo minicom -s

Désactiver les contrôles de flux logiciels et matériels
Baudrate : 115200

%Penser à sauvegarde la configuration pour ne pas avoir à la re-entrer à chaque fois

Pour quitter minicom : ctrl + A puis q


\item PuTTy : ubuntu : 
\begin{minted}{bash}
sudo apt-get install putty
\end{minted}
Choisir le mode série, mettre le port de la carte, baudrate 115200

\end{enumerate}

%Envoyer des commandes AT à la carte :



\section{Lire les messages envoyés sur LoRa server}

Obtenir des identifiants LoRa server, contacter : didier.donsez@univ-grenoble-alpes.fr



\subsection{Avec l'exemple End Node  du software demo -cube-lrwan (URL plus haut)}

\subsubsection{Etape 1 : Flasher le programme sur la carte}

\begin{enumerate}
    


\item Ouvrir le projet avec STM32Cube IDE \\ chemin :  /home/username/Documents/STM32CubeExpansion\_LRWAN
\_V1.2.1/Projects/B-L072Z-LRWAN1/Applications/LoRa \\
en utilisant : File, Open project from filesystem
Puis bouton build all \\
\item Trouver l'emplacement du fichier .bin généré : 
\begin{minted}{bash}
find /home | grep .bin$
\end{minted}.elf, le fichier .bin est dans le même dossier que le .elf

\item Ouvrir STM32Cube Programmer :
Connecter la carte cf Flasher le FW avec STM32 cube programmer
Flasher le .bin
Déconnecter la carte avec le bouton disconnect
Ouvrir PuTTy et connecter la carte cf  Utiliser  un  terminal  série  pour  commandes
AT et communiquer avec la carte
Appuyer sur le bouton reset de la carte 

Le DevUI de la carte et l'App Key devraient s'afficher.

\end{enumerate}


\subsubsection{Etape 2 : Enregitrer la carte sur LoRa server et établir la connexion entre le serveur et la carte}

\begin{enumerate}
    

 \item Enregistrer la carte sur le réseau LoRaWAN :
tuto 
\url{https://github.com/CampusIoT/tutorial/blob/master/loraserver/README-app.md}

\item Créer une application bouton Applications à gauche
Champs Nom, description : pas d'importance.
Service profile :choisir DEFAULT

\item Créer un device dans l'application
Entrer le DEVUI de la carte
Device profile : CLASS\_A\_OTAA

\item Entrer l'app key dans la clé KEY(OTAA)

\item Appuyer sur Reset de la carte

\item La carte va se connecter et les trames sont visibles depuis l'onglet Live LoRaWan Frames

info envoyées : Coordonnées GPS
\end{enumerate}

%Dans le fichier Commissioning.h :

%Renseigner le champ LORAWAN\_NETWORK\_ID, i'ID s'obtient sur le site loraserver

%Bien penser à rebuilder l'appli puis re-télécharger le FW sur la carte.

%Activer la carte


%\subsection{Obtenir une appkey et un Deveui : cas du STM32 LoRa and Sigfox discovery kit}
%D'après le feuillet :





%\section{Mettre en plce une gateway pycom LoPy 4}

%Suivre le tuto : https://github.com/CampusIoT/tutorial/blob/master/loraserver/README-gateway.md






\end{document}

