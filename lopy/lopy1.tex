\documentclass{article}
\usepackage[utf8]{inputenc}
\usepackage{minted}
\usepackage{graphicx}

\title{pycom lopy4}
\author{cmonaton }
\date{July 2019}

\begin{document}











\maketitle

\section{Se connecter à la carte pour la programmer}

\subsection{Utiliser Telnet pour se connecter à la carte}

Tuto sur ce lien : https://docs.pycom.io/gettingstarted/programming/repl/telnet/

Double cliquer sur l'icône wifi du pycom lopy pour s'y connecter, dans mon cas lopy-wlan-892c

Ouvrir un terminal et taper : telnet 192.168.4.1 \\
login : micro \\
mdp : python \\

\subsubsection{Programmer la carte depuis cette interface}


\begin{enumerate}
    \item Afficher "Hello" : 
    \begin{minted}{python}
variable = "Hello World"
print(variable) 
    \end{minted}

    \item Allumer la led en vert : 
    
 \begin{minted}{python}
import pycom
pycom.heartbeat(False)
pycom.rgbled(0xff00)          
    \end{minted}
\end{enumerate}

Pour coller dans un terminal : Maj + Ctrl + V

\subsection{Se connecter à la carte par liaison série}
Si le firmware de base est modifié, on ne peut plus se connecter en Wi-Fi à la carte. 
Dans ce cas, on peut utiliser la liason usb avec Atom et pymark

\subsubsection{installer Atom text editor}

\begin{minted}{bash}





sudo add-apt-repository ppa:webupd8team/atom
sudo apt install atom


\end{minted}
atom pour lancer l'éditeur

\subsubsection{install pymark}
tuto : https://docs.pycom.io/pymakr/installation/atom/

Depuis atom selon l'image installer pymakr


    \begin{figure}[H]
\begin{center}
\advance\leftskip-3cm
\advance\rightskip-3cm
\includegraphics[keepaspectratio=true,scale=0.3]{atom2.png}
\label{visina8}
\end{center}\end{figure}

\subsubsection{Connexion à la carte}

Déterminer le port séie sur lequel la carte et montée :

Après le branchement : 

\begin{minted}{bash}
dmesg | grep tty 
\end{minted}

ttyACM0 dans mon cas

\begin{minted}{bash}

sudo chmod 666 /dev/ttyACM0

\end{minted}
Selon l'image utiliser la console d'Atom pour se connecter à la carte :

  \begin{figure}[H]
\begin{center}
\advance\leftskip-3cm
\advance\rightskip-3cm
\includegraphics[keepaspectratio=true,scale=0.3]{atom3.png}
\label{visina8}
\end{center}\end{figure}

\subsection{Avec un terminal série type PuTTY}

PuTTY : sélectionner liaison série, choisir le bon port /dev/ttyACM0 dans mon cas, baudrate 115200.




\end{document}

